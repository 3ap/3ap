%%%%%%%%%%%%%%%%%%%%%%%%%%%%%%%%%%%%%%%%%
% Medium Length Professional CV
% LaTeX Template
% Version 2.0 (8/5/13)
%
% This template has been downloaded from:
% http://www.LaTeXTemplates.com
%
% Original author:
% Trey Hunner (http://www.treyhunner.com/)
%
% Important note:
% This template requires the resume.cls file to be in the same directory as the
% .tex file. The resume.cls file provides the resume style used for structuring the
% document.
%
%%%%%%%%%%%%%%%%%%%%%%%%%%%%%%%%%%%%%%%%%

%----------------------------------------------------------------------------------------
%	PACKAGES AND OTHER DOCUMENT CONFIGURATIONS
%----------------------------------------------------------------------------------------

\documentclass[hidelinks]{resume} % Use the custom resume.cls style
\usepackage{fontspec}% provides font selecting commands
\usepackage{xunicode}% provides unicode character macros
\usepackage{xltxtra} % provides some fixes/extras
\usepackage{setspace}
\setmainfont{Roboto}

\usepackage[left=0.35in,top=0.2in,right=0.75in,bottom=0.2in]{geometry} % Document margins
\usepackage{hyperref}
\usepackage{graphicx}
\graphicspath{ {./images/} }

\usepackage{xcolor}
\hypersetup{
    colorlinks=true,
    linkcolor={red!50!black},
    citecolor={blue!50!black},
    urlcolor={blue!80!black},
    linkbordercolor={blue!80!black}
}

% Обязательно переносить слова, чтобы соблюсти поля документа. Для
% соблюдения полей можно пренебречь правилами для тех слов и
% словосочетаний, о которых не знают словаря переносов (ruhyphen или
% ruenhyph). Оно почему-то работает. Взято с:
%
%   http://www.latex-community.org/forum/viewtopic.php?p=70342#p70342
%
\tolerance 1414
\hbadness 1414
\emergencystretch 1.5em
\hfuzz 0.3pt
\widowpenalty=10000
\vfuzz \hfuzz
\raggedbottom

\begin{document}

%\begin{minipage}[c]{0.18\textwidth}
%      \includegraphics[width=0.9\textwidth]{portrait}
%\end{minipage}%
%\begin{minipage}[c]{0.82\textwidth}
{\large \bf Сергей Назарьев,
\href{mailto:sergey@nazaryev.ru}{sergey@nazaryev.ru} \vspace{2mm}} \\
{\small Я хорошо разбираюсь в том, как работают и взаимодействуют
основные части Linux'а (процесс загрузки, система инициализации,
супервайзинг, сеть, различные виды IPC и др.), понимаю принципы
работы сетей, умею работать с системами контроля версий, с
баг-трекерами и всем, что требует сейчас от разработчика индустрия.}
%\end{minipage}

%----------------------------------------------------------------------------------------
%	TECHNICAL STRENGTHS SECTION
%----------------------------------------------------------------------------------------

\begin{rSection}{Навыки (подчёркнуты основные навыки)}
{\setstretch{1.3}
\begin{tabular}{ @{} >{\bfseries}l @{\hspace{6ex}} l }
Активно программирую на& \underline{C} и \underline{POSIX shell} \\
Имел опыт программирования на& Python, C\#, Java, PHP, Lua \\
Системы контроля версий& Git, Subversion \\
Сборка& \underline{GNU make}, \underline{Jenkins}, GitLab CI\\
Дистрибутивы& \underline{Debian}, \underline{Arch Linux}, \underline{Buildroot}, Yocto\\
Языки разметки& \underline{Markdown}, \LaTeX, HTML5/CSS3 \\
Другое& \underline{vim}, docker\\
Владение языками& русский, английский (чтение технической литературы)\\
\end{tabular}
}

\end{rSection}

%----------------------------------------------------------------------------------------
%	EDUCATION SECTION
%----------------------------------------------------------------------------------------

\begin{rSection}{Образование}

{\bf Университет ИТМО (СПбНИУ ИТМО)} \hfill {\em сентябрь 2013 -- июль 2017} \\
Бакалавр. Информатика и вычислительная техника
\end{rSection}

%----------------------------------------------------------------------------------------
%	WORK EXPERIENCE SECTION
%----------------------------------------------------------------------------------------

\begin{rSection}{Опыт работы (3+ года опыта)}

%------------------------------------------------

\begin{rSubsection}{\href{http://protei-st.ru}{Протей СТ}, решения для телекоммуникационных сетей}{\em июль 2016 -- сентябрь 2018 (2 года 2 месяца)}{Инженер Linux (embedded)}{Санкт-Петербург}
\item Принимал участие в проектировании и разработке in-house дистрибутива на основе Buildroot для созданных компанией устройств (x86, ARM);
\item Портировал и дорабатывал драйвера, в редких случаях писал небольшие драйвера для Linux и U-Boot с нуля;
\item Активно помогал «железячникам», QA и прикладным программистам в локализации и исправлении программно-аппаратных багов;
\item Оптимизировал процесс разворачивания ОС и прикладного ПО на новые устройства;
\item Адаптировал российские дистрибутивы (Astra Linux, МСВС) для их корректной работы на устройствах, разработанных компанией;
\item Внедрил в компании единые правила автоматизации сборки с помощью Jenkins, GitLab CI и Docker;
\item Консультировал прикладных программистов и принимал участие в разработке тех. процесса подготовки исходных текстов для прохождения сертификации ФСТЭК, МО РФ.
\end{rSubsection}

%------------------------------------------------

\begin{rSubsection}{\href{http://ntc.metrotek.ru}{НТЦ Метротек}, сетевое оборудование}{\em февраль 2015 -- июнь 2016 (1 год 5 месяцев)}{Системный программист Linux (embedded)}{Санкт-Петербург}
\item Успешно спортировал прикладное ПО с RTOS (AVR) на GNU/Linux (ARM);
\item Доработал Debian GNU/Linux для запуска на приборах производства компании;
\item Внедрил в компании автоматизацию «чистой сборки» ПО с помощью Jenkins и Puppet;
\item Помогал в пакетировании прикладного ПО под Debian GNU/Linux.
\end{rSubsection}

%------------------------------------------------

\begin{rSubsection}{\href{http://mobiumapps.com}{Mobium}, генератор мобильных приложений}{\em февраль 2014 --  август 2014 (7 месяцев)}{Android-разработчик}{Санкт-Петербург}
\item Дорабатывал Android-приложение для компании 220 Вольт;
\item Дорабатывал панель управления на ASP.NET MVC.
\end{rSubsection}

%------------------------------------------------

\begin{rSubsection}{Firepush, стартап}{\em август 2013 -- февраль 2014 (7 месяцев)}{Android-разработчик, part-time}{Санкт-Петербург}
\item Разработал генератор типовых Android-приложений на Python + PyQt;
\item Разработал модуль показа рекламы для встраивания в third-party приложения Android.
\end{rSubsection}

%------------------------------------------------

\begin{rSubsection}{\href{http://nopreset.ru}{Nopreset}, web-студия}{\em май 2012 -- сентябрь 2012 (4 месяца)}{PHP-программист, летняя стажировка}{Саратов}
\item Разработал backend сайтов на базе CMF MODx Revo/Evo (PHP): \href{http://cpt-yurcom.ru}{cpt-yurcom.ru}, \href{http://okean-tur.ru}{okean-tur.ru}, \href{http://pioner-kino.ru}{pioner-kino.ru}
\end{rSubsection}
\end{rSection}

%----------------------------------------------------------------------------------------

\end{document}
