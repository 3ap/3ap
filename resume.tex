%%%%%%%%%%%%%%%%%%%%%%%%%%%%%%%%%%%%%%%%%
% Medium Length Professional CV
% LaTeX Template
% Version 2.0 (8/5/13)
%
% This template has been downloaded from:
% http://www.LaTeXTemplates.com
%
% Original author:
% Trey Hunner (http://www.treyhunner.com/)
%
% Important note:
% This template requires the resume.cls file to be in the same directory as the
% .tex file. The resume.cls file provides the resume style used for structuring the
% document.
%
%%%%%%%%%%%%%%%%%%%%%%%%%%%%%%%%%%%%%%%%%

%----------------------------------------------------------------------------------------
%	PACKAGES AND OTHER DOCUMENT CONFIGURATIONS
%----------------------------------------------------------------------------------------

\documentclass[hidelinks]{resume} % Use the custom resume.cls style
\usepackage[utf8]{inputenc}
\usepackage[english,russian]{babel}
\usepackage{cmap}

\renewcommand{\familydefault}{\sfdefault}

\usepackage[left=0.75in,top=0.3in,right=0.75in,bottom=0.1in]{geometry} % Document margins
\usepackage{hyperref}

\usepackage{xcolor}
\hypersetup{
    colorlinks=true,
    linkcolor={red!50!black},
    citecolor={blue!50!black},
    urlcolor={blue!80!black},
    linkbordercolor={blue!80!black}
}

\name{Сергей Назарьев, 21 год} % Your name

\begin{document}

\begin{tabular}{ @{} >{\bfseries}l @{\hspace{6ex}} l }
Личный сайт& \href{http://nazaryev.ru}{nazaryev.ru} \\
Электронная почта& \href{mailto:sergey@nazaryev.ru}{sergey@nazaryev.ru} \\
Контактный телефон& +7 (931) 225-74-28
\end{tabular}

%----------------------------------------------------------------------------------------
%	TECHNICAL STRENGTHS SECTION
%----------------------------------------------------------------------------------------

\begin{rSection}{Навыки}
\begin{tabular}{ @{} >{\bfseries}l @{\hspace{6ex}} l }
Языки программирования& \underline{C}, \underline{POSIX shell}, Python, C\#, Java, PHP, Lua, x86 Assembly \\
Системы контроля версий& Git, Subversion \\
Сборка& \underline{GNU make}, \underline{Buildroot}, GNU Autotools, Jenkins, GitLab CI\\
Дистрибутивы& \underline{Debian}, \underline{Arch Linux}\\
Языки разметки& \underline{Markdown}, \LaTeX, HTML5/CSS3 \\
Серверное ПО& nginx, Apache \\
Другое ПО& \underline{vim}, strace, gdb, Puppet\\
Дополнительные навыки& чувство вкуса, любовь к поиску уязвимостей\\
Владение языками& русский, технический английский\\
\end{tabular}

\end{rSection}

%----------------------------------------------------------------------------------------
%	EDUCATION SECTION
%----------------------------------------------------------------------------------------

\begin{rSection}{Образование}

{\bf Университет ИТМО (СПбНИУ ИТМО)} \hfill {\em Сентябрь 2013 - Июль 2017} \\
Бакалавр. Информатика и вычислительная техника
\end{rSection}

%----------------------------------------------------------------------------------------
%	WORK EXPERIENCE SECTION
%----------------------------------------------------------------------------------------

\begin{rSection}{Опыт работы}

%------------------------------------------------

\begin{rSubsection}{\href{http://protei-st.ru}{НТЦ Протей}, решения для телекоммуникационных сетей}{\em Июль 2016 - настоящее время}{Embedded Linux BSP Developer}{Санкт-Петербург}
\item Разработка дистрибутива на основе Buildroot для устройств собственной разработки;
\item Портирование, правка и разработка драйверов Linux;
\item Доработка и конфигурирование U-Boot для ARM (i.MX6);
\item Автоматизация сборки с помощью GitLab CI и Docker;
\item Адаптация дистрибутивов (Astra Linux, МСВС) для существующих и новых аппаратных платформ;
\item Оптимизация процесса разворачивания прошивок (ОС и ПО) на новые устройства;
\item Помощь в отладке и устранении программно-аппаратных багов.
\end{rSubsection}

%------------------------------------------------

\begin{rSubsection}{\href{http://metrotek.spb.ru}{НТЦ Метротек}, сетевое оборудование}{\em Февраль 2015 - Июнь 2016}{Системный программист Linux}{Санкт-Петербург}
\item Разработка системного ПО на С для взаимодействия с FPGA (Cyclone 5) и периферией;
\item Разработка прикладного ПО на С для тестирования сетевого оборудования (RFC 2544, Y.1564);
\item Портирование программного обеспечения с bare metal (AVR) на GNU/Linux (ARM);
\item Разработка BSP на основе Debian GNU/Linux для приборов собственной разработки;
\item Модификация прошивки Android для измерительного прибора;
\item Организация «чистой сборки» и пакетирования ПО под Debian GNU/Linux;
\item Автоматизация сборки с помощью Jenkins и Puppet.
\end{rSubsection}

%------------------------------------------------

\begin{rSubsection}{\href{http://mobiumapps.com}{Mobium}, генератор мобильных приложений}{\em Февраль 2014 - Сентябрь
2014}{Android-разработчик}{Санкт-Петербург}
\item Разработка ПО на заказ под Android;
\item Разработка панели управления на ASP.NET MVC.
\end{rSubsection}

%------------------------------------------------

\begin{rSubsection}{Firepush, стартап}{\em Август 2013 - Февраль 2014}{Программист}{Санкт-Петербург}
\item Разработка генератора приложений под Android;
\item Разработка модуля рекламы для встраивания в приложения Android;
\item Aдминистрирование серверов на Debian GNU/Linux.
\end{rSubsection}

%------------------------------------------------

\begin{rSubsection}{\href{http://nopreset.ru}{Nopreset}, web-студия}{\em Май 2012 - Сентябрь 2012}{PHP-программист, стажёр}{Саратов}
\item Разработка backend на базе CMF MODx Revo/Evo (PHP).
\end{rSubsection}
\end{rSection}

%----------------------------------------------------------------------------------------

\end{document}
